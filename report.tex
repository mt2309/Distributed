\documentclass[11pt]{amsart}
\usepackage{geometry}                % See geometry.pdf to learn the layout options. There are lots.
\geometry{letterpaper}                   % ... or a4paper or a5paper or ... 
%\geometry{landscape}                % Activate for for rotated page geometry
%\usepackage[parfill]{parskip}    % Activate to begin paragraphs with an empty line rather than an indent
\usepackage{graphicx}
\usepackage{amssymb}
\usepackage{epstopdf}
\DeclareGraphicsRule{.tif}{png}{.png}{`convert #1 `dirname #1`/`basename #1 .tif`.png}

\title{Parallel Algorithms Coursework}
\author{Fraser Waters and Michael Thorpe}
%\date{}                                           % Activate to display a given date or no date

\begin{document}
\maketitle

\section{Failure detectors}

Failure detectors maintain a collection of suspected processes. A suspected
process is one that is suspected of being in a crashed state at the time the
failure detector is evaluated.

Detectors in different processes do not have to agree and detectors do not have
to be correct, that is a detector may suspect a correct process or not suspect
a crashed process.

Detectors can be classified by their completeness and accuracy. A complete
detector will suspect crashed processes.  An accurate detector will not detect
correct processes. A trivial complete detector is a detector that suspects
every process. A trivial accurate detector is one that suspect no processes.
Clearly neither of these are very useful or interesting, and a trade-off
between accuracy and completeness is more useful.

A detector is strongly complete if every crashed process is eventually
suspected by every correct process. A detector is strongly accurate if no
correct process is ever suspected, it is weakly accurate if there is at least
one correct process that is never suspected. A detector is eventually strongly
accurate if it eventually no longer suspects any correct process, a detector is
eventually weakly accurate if eventually there is a correct process that is not
suspected. 

\section{Implementations}

\subsection{Perfect failure detector}

The implementation of the perfect failure detector assumes that no message is
delayed for more than two times the average delay. As long as we wait for at
least $\text{time between heartbeats} + 2 * delay$ we will never incorrectly
suspect a process. 

\subsection{Eventually perfect failure detector}

The eventually perfect failure detector keeps track of the maximum delay
between heartbeats from each process. If a process heartbeat is not received
within two times the maximum delay tracked so far it is suspected.

\section{Relationships between classes}

\begin{description}
	\item[$P$ (perfect)] Strongly complete and strongly accurate
	\item[$S$ (strong)] Strongly complete and weakly accurate
	\item[$\diamond P$ (eventually perfect)] Strongly complete and eventually strongly accurate
	\item[$\diamond S$ (eventually strong)] Strongly complete and eventually weakly accurate
\end{description}

$P$ emulates $S$. 
$\diamond P$ emulates $\diamond S$, as it is eventually strongly accurate which
is more specific then $\diamond S$ that is only eventually weakly accurate.
$S$ emulates $\diamond S$ as is is weakly accurate immediately, this is equivalent to $\diamond S$ eventually time being 0.
$S$ can not emulate $\diamond P$ as $\diamond P$ is strongly accurate, which is more specific than $S$.
$\diamond S$ can not emulate any of the others, it is the weakest failure detector.

\end{document}  
