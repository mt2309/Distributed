\documentclass[11pt]{amsart}
\usepackage{geometry}                % See geometry.pdf to learn the layout options. There are lots.
\geometry{letterpaper}                   % ... or a4paper or a5paper or ... 
%\geometry{landscape}                % Activate for for rotated page geometry
%\usepackage[parfill]{parskip}    % Activate to begin paragraphs with an empty line rather than an indent
\usepackage{graphicx}
\usepackage{amssymb}
\usepackage{epstopdf}
\DeclareGraphicsRule{.tif}{png}{.png}{`convert #1 `dirname #1`/`basename #1 .tif`.png}

\title{Parallel Algorithms Coursework}
\author{Fraser Waters and Michael Thorpe}
%\date{}                                           % Activate to display a given date or no date

\begin{document}
\maketitle

\section{Failure detectors}

Failure detectors maintain a collection of suspected processes. A suspected
process is one that is suspected of being in a crashed state at the time the
failure detector is evaluated.

Detectors in different processes do not have to agree and detectors do not have
to be correct, that is a detector may suspect a correct process or not suspect
a crashed process.

Detectors can be classified by their completeness and accuracy. A complete
detector will suspect crashed processes.  An accurate detector will not detect
correct processes. A trivial complete detector is a detector that suspects
every process. A trivial accurate detector is one that suspect no processes.
Clearly neither of these are very useful or interesting, and a trade-off
between accuracy and completeness is more useful.







%\subsection{}


\end{document}  
